\documentclass[12pt,nocut]{article}

\usepackage{latex_style/packages}
\usepackage{latex_style/notations}

\title{Spin models and ML}
\date{}

\begin{document}

\maketitle{}

\section{Intro}
Bla

\section{Bayesian inference with Gaussian additive noise}\label{sec:i_mmse}

We start with the following model:
\begin{equation}\label{eq:inference}
	\bbf{Y} = \sqrt{\lambda} \, \bbf{X} + \bbf{Z} \,,
\end{equation}
where the signal $\bbf{X}$ is sampled according to some probability distribution $P_X$ over $\R^n$, and where the noise $\bbf{Z} =(Z_1, \dots, Z_n) \iid \cN(0,1)$ is independent from $\bbf{X}$. 
The parameter $\lambda \geq 0$ plays the role of a signal-to-noise ratio.
We assume that $P_X$ admits a finite second moment: $\E \|\bbf{X}\|^2 < \infty$. No other assumption is made on $P_X$. In particular, the fact that the noise is i.i.d. in the dimension $n$ will be crucial in the computation below but no such assumption is made on the signal which is allowed to have any complicated structure component-wise.

Given the observation channel~\eqref{eq:inference}, the goal of the statistician is to estimate $\bbf{X}$ given the observations $\bbf{Y}$. We assume to be in the ``Bayes-optimal'' setting, where the statistician knows all the parameters of the inference model, that is the prior distribution $P_X$ and the signal-to-noise ratio $\lambda$. 
As will be clear below, the Mean Squared Error (MSE) is the natural measure of performance when dealing with Gaussian noise. In this section, we measure the performance of an estimator $\widehat{\theta}$ (i.e.\ a measurable function of the observations $\bbf{Y}$) by its Mean Squared Error $\MSE(\widehat{\theta}) = \E \| \bbf{X} - \widehat{\theta}(\bbf{Y}) \|^2$.
%One of our main quantity of interest will be 
\begin{definition}\label{def:MMSE}
For a signal $\bbf{X}$ and observations $\bbf{Y}$, the Minimum Mean Squared Error is defined by
$$
\MMSE(\lambda) \defeq \min_{\widehat{\theta}} \MSE(\widehat{\theta})= \E \left[ \left\| \bbf{X} - \E[\bbf{X}|\bbf{Y}] \right\|^2 \right],
$$
where the minimum is taken over all measurable function $\widehat{\theta}$ of the observations $\bbf{Y}$. 
The optimal estimator (in term of Mean Squared Error) is the posterior mean of the signal $\bbf{X}$ given $\bbf{Y}$.
\end{definition}
\begin{proof}
Using $\E[\bbf{X}^T\widehat{\theta}(\bbf{Y})|\bbf{Y}]  = \E[\bbf{X}|\bbf{Y}]^T \widehat{\theta}(\bbf{Y})$, we have
\begin{align*}
\E \left[ \left\| \bbf{X} -  \widehat{\theta}(\bbf{Y})\right\|^2 \right] & = \E\left[ \left\| \bbf{X}\right\|^2\right] - 2 \E \left[ \E[\bbf{X}|\bbf{Y}]^T \widehat{\theta}(\bbf{Y})\right] + \E\left[ \left\| \widehat{\theta}(\bbf{Y})\right\|^2\right]
\end{align*}
and 
\begin{align*}
\E \left[ \left\| \E[\bbf{X}|\bbf{Y}] -  \widehat{\theta}(\bbf{Y})\right\|^2 \right] & = \E\left[ \left\| \E[\bbf{X}|\bbf{Y}]\right\|^2\right] - 2 \E \left[ \E[\bbf{X}|\bbf{Y}]^T \widehat{\theta}(\bbf{Y})\right] + \E\left[ \left\| \widehat{\theta}(\bbf{Y})\right\|^2\right]\geq 0,
\end{align*}
we deduce
\begin{align*}
\E \left[ \left\| \bbf{X} -  \widehat{\theta}(\bbf{Y})\right\|^2 \right] & \geq \E\left[ \left\| \bbf{X}\right\|^2\right] - \E\left[ \left\| \E[\bbf{X}|\bbf{Y}]\right\|^2\right]\\
& = \E\left[ \left\| \bbf{X}\right\|^2\right] -2\E \left[ \E[\bbf{X}|\bbf{Y}]^T\E[\bbf{X}|\bbf{Y}] \right] + \E\left[ \left\| \E[\bbf{X}|\bbf{Y}]\right\|^2\right]\\
& = \E \left[ \left\| \bbf{X} - \E[\bbf{X}|\bbf{Y}] \right\|^2 \right]
\end{align*}
\end{proof}

Following Definition \ref{def:MMSE}, we see that a natural object to study is the posterior distribution of $\bbf{X}$.
We have for the joint distribution:
$$
dP_{(X,Y)}(\bbf{x},\bbf{y}) \propto dP_X(\bbf{x}) e^{\frac{\| \bbf{y}-\sqrt{\lambda}\bbf{x}\|^2}{2}}d\bbf{y}
$$
By Bayes rule, the posterior distribution of $\bbf{X}$ given $\bbf{Y}$ is
\begin{equation}\label{eq:posterior_0}
	dP( \bbf{x} \, | \, \bbf{Y}  ) 
	= \frac{1}{\cZ(\lambda,\bbf{Y})} e^{H_{\lambda,\bbf{Y}}(\bbf{x})} dP_X(\bbf{x}) \,,
\end{equation}
where 
$$
H_{\lambda,\bbf{Y}}(\bbf{x})=
\sqrt{\lambda}\, \bbf{x}^{\sT} \bbf{Y} - \frac{\lambda}{2} \| \bbf{x} \|^2
=
\sqrt{\lambda} \, \bbf{x}^{\sT} \bbf{Z} + \lambda \, \bbf{x}^{\sT} \bbf{X} - \frac{\lambda}{2} \| \bbf{x} \|^2 \,.
$$
\begin{definition}
	$H_{\lambda,\bbf{Y}}$ is called the Hamiltonian\protect\footnotemark and the normalizing constant
	$$
	\cZ(\lambda,\bbf{Y}) = \int dP_X(\bbf{x}) e^{H_{\lambda,\bbf{Y}}(\bbf{x})}
	$$
	is called the partition function. 
\end{definition}
\footnotetext{According to the physics convention, this should be minus the Hamiltonian, since a physical system tries to minimize its energy. However, we chose here to remove it for simplicity.}
Expectations with respect the posterior distribution~\eqref{eq:posterior_0} will be denoted by the Gibbs brackets $\langle \cdot \rangle_{\lambda}$:
$$
\big\langle f(\bbf{x}) \big\rangle_{\lambda}
= \E \big[f(\bbf{X})|\bbf{Y} \big]
= \frac{1}{\cZ(\lambda,\bbf{Y})} \int dP_X(\bbf{x}) f(\bbf{x}) e^{H_{\lambda,\bbf{Y}}(\bbf{x})} \,,
$$
for any measurable function $f$ such that $f(\bbf{X})$ is integrable.

\begin{definition}
	$F(\lambda) = \E \log \cZ(\lambda,\bbf{Y})$ is called the free energy\footnotemark. It is related to the mutual information between $\bbf{X}$ and $\bbf{Y}$ by
	\begin{equation}\label{eq:f_i}
		F(\lambda) = \frac{\lambda}{2} \E \| \bbf{X} \|^2 - I(\bbf{X};\bbf{Y})\,.
	\end{equation}
\end{definition}
\footnotetext{This is in fact minus the free energy, but we chose to remove the minus sign for simplicity.}
\begin{proof}
	The mutual information $I(\bbf{X};\bbf{Y})$ is defined as the Kullback-Leibler divergence between $P_{(X,Y)}$, the joint distribution of $(\bbf{X},\bbf{Y})$ and $P_X \! \otimes \! P_Y$ the product of the marginal distributions of $\bbf{X}$ and $\bbf{Y}$. $P_{(X,Y)}$ is absolutely continuous with respect to $P_X \otimes P_Y$ with Radon-Nikodym derivative:
	$$
	\frac{d P_{(X,Y)}}{d P_X \! \otimes \! P_Y}(\bbf{X},\bbf{Y})
	=
	\frac{\exp\left(-\frac{1}{2}\| \bbf{Y} - \sqrt{\lambda} \bbf{X} \|^2\right)}{\int \exp\left(-\frac{1}{2}\| \bbf{Y} - \sqrt{\lambda} \bbf{x} \|^2\right) dP_X(\bbf{x})} \,.
	$$
	Therefore
	\begin{align*}
		I(\bbf{X};\bbf{Y}) 
		&= \E \log\left( \frac{d P_{(X,Y)}}{d P_X \!\otimes \!P_Y}(\bbf{X},\bbf{Y}) \right)\\
		&= - \E \log \int dP_X(\bbf{x}) \exp \left( \sqrt{\lambda} \bbf{x}^{\sT} \bbf{Y} - \sqrt{\lambda} \bbf{X}^{\sT} \bbf{Y} - \frac{\lambda}{2} \|\bbf{x}\|^2 + \frac{\lambda}{2}\|\bbf{X}\|^2 \right)
		\\
                &= -F(\lambda) +\sqrt{\lambda}\E \bbf{X}^T\bbf{Y} - \frac{\lambda}{2} \E \|\bbf{X} \|^2\\
		&= - F(\lambda) + \frac{\lambda}{2} \E \| \bbf{X} \|^2 \,,
	\end{align*}
since $\sqrt{\lambda}\bbf{X}^T\bbf{Y} = \lambda \|\bbf{X}\|^2 + \sqrt{\lambda}\bbf{X}^T\bbf{Z}$.
\end{proof}

We state now two basic properties of the $\MMSE$. A more detailed analysis can be found in~\cite{guo2011estimation,wu2012functional}.
\begin{proposition}\label{prop:mmse_dec}
	$\lambda \mapsto \MMSE(\lambda)$ is non-increasing over $\R_{\geq 0}$. 
	%If $P_X$ is not a Dirac mass, then $\MMSE$ is strictly decreasing on $\R_{\geq 0}$.
	Moreover
	\begin{itemize}
		\item $\MMSE(0) = \E\|\bbf{X} - \E[\bbf{X}]\|^2$,
		\item $\MMSE(\lambda) \xrightarrow[\lambda \to +\infty]{} 0$.
	\end{itemize}
\end{proposition}
\begin{proof}
Let $0 < \lambda_2 < \lambda_1$. Define $\Delta_1 = \lambda_1^{-1}$, $\Delta_2 = \lambda_2^{-1}$ and
	$$
	\begin{cases}
		\bbf{Y}_1 = \bbf{X} + \sqrt{\Delta_1} \bbf{Z}_1 \\
		\bbf{Y}_2 = \bbf{X} + \sqrt{\Delta_1} \bbf{Z}_1  + \sqrt{\Delta_2 - \Delta_1} \bbf{Z}_2 \,,
	\end{cases}
	$$
	where $\bbf{X} \sim P_X$ is independent from $\bbf{Z}_1,\bbf{Z}_2 \iid \cN(0,\Id_n)$. Now, by independence between $(\bbf{X},\bbf{Y}_1)$ and $\bbf{Z}_2$ we have
	\begin{align*}
		\MMSE(\lambda_1) 
		&= \E \left\| \bbf{X} - \E [\bbf{X} | \bbf{Y}_1] \right\|^2
		= \E \left\| \bbf{X} - \E [\bbf{X} | \bbf{Y}_1, \bbf{Z}_2] \right\|^2
		= \E \left\| \bbf{X} - \E [\bbf{X} | \bbf{Y}_1, \bbf{Y}_2] \right\|^2
		\\
		&\leq \E \left\| \bbf{X} - \E [\bbf{X} | \bbf{Y}_2] \right\|^2
		= \MMSE(\lambda_2) \,.
	\end{align*}
	Next, notice that
	\begin{equation}\label{eq:upper_bound_mmse}
		\MMSE(\lambda_1)
		= \E \left\| \bbf{X} - \E [\bbf{X} | \bbf{Y}_1] \right\|^2
		\leq \E \left\| \bbf{X} - \E [\bbf{X}] \right\|^2 = \MMSE(0)\,.
	\end{equation}
	This shows that the $\MMSE$ is non-increasing on $\R_{\geq 0}$. 
	The first point is obvious while the second follows from:
	$$
	0 \leq \MMSE(\lambda)
	= \E \| \bbf{X} - \E[\bbf{X}| \bbf{Y}]\|^2
	\leq \E \| \bbf{X} - \frac{1}{\sqrt{\lambda}} \bbf{Y}\|^2 = \frac{n}{\lambda} \xrightarrow[\lambda \to +\infty]{} 0 \,.
	$$
\end{proof}
\begin{proposition}\label{prop:mmse_cont}
	$\lambda \mapsto \MMSE(\lambda)$ is continuous over $\R_{\geq 0}$.
\end{proposition}
The proof of Proposition~\ref{prop:mmse_cont} can be found in Appendix~\ref{sec:proof_mmse_cont}.
\\

We present now the very useful ``I-MMSE'' relation from~\cite{guo2005mutual}. This relation was previously known (under a different formulation) as ``de Brujin identity'' see \cite[Equation 2.12]{stam1959some}.
\begin{proposition}\label{prop:i_mmse}
	For all $\lambda \geq 0$,
	\begin{equation}\label{eq:i_mmse}
		\frac{\partial}{\partial \lambda} I(\bbf{X};\bbf{Y}) = \frac{1}{2} \MMSE(\lambda)
		\qquad \text{and} \qquad
		F'(\lambda) = \frac{1}{2} \E \langle \bbf{x}^{\sT} \bbf{X} \rangle_{\lambda} 
		= \frac{1}{2} \big(\E \| \bbf{X}\|^2 - \MMSE(\lambda) \big)
		\,.
	\end{equation}
	$F$ thus is a convex, differentiable, non-decreasing, and $\frac{1}{2}\E\|\bbf{X}\|^2$-Lipschitz function over $\R_{\geq 0}$. If $P_X$ is not a Dirac mass, then $F$ is strictly convex.
\end{proposition}

TBC...

\newpage
Proposition~\ref{prop:i_mmse} is proved in Appendix~\ref{sec:proof_i_mmse}.
Proposition~\ref{prop:i_mmse} reduces the computation of the MMSE to the computation of the free energy. This will be particularly useful because the free energy $F$ is much easier to handle than the MMSE.

We end this section with the simplest model of the form~\eqref{eq:inference}, namely the additive Gaussian scalar channel:
\begin{equation}\label{eq:additive_scalar_channel}
	Y = \sqrt{\lambda} X + Z \,,
\end{equation}
where $Z \sim \cN(0,1)$ and $X$ is sampled from a distribution $P_0$ over $\R$, independently of $Z$. The corresponding free energy and the MMSE are respectively
	\begin{equation}\label{eq:def_psi_p0}
		\psi_{P_0}(\lambda)= \E \log \int dP_0(x)e^{\sqrt{\lambda} \,Y x - \lambda x^2/2}
		\quad \text{and} \quad \MMSE_{P_0}(\lambda) = \E \big[\big(X - \E[X|Y]\big)^2\big] \,.
	\end{equation}
The study of this simple inference channel will be very useful in the following, because we will see that the inference problems that we are going to study enjoy asymptotically a ``decoupling principle'' that reduces them to scalar channels like~\eqref{eq:additive_scalar_channel}.
%The next proposition summarizes the main properties of the free energy $\psi_{P_0}$ of the channel~\eqref{eq:additive_scalar_channel}. 

%\begin{proposition}\label{prop:additive_scalar_channel}
	%Let $X \sim P_0$ be a real random variable with finite second moment. For $\lambda \geq 0$, let $Y = \sqrt{\lambda} X + Z$, where $Z \sim \cN(0,1)$ is independent from $X_0$.
	%Then the function
	%\begin{equation}\label{eq:def_psi_p0}
		%\psi_{P_0}: \lambda \mapsto \E \log \int dP_0(x)e^{\sqrt{\lambda} \,Y x - \lambda x^2/2}
	%\end{equation}
	%is convex, continuously differentiable, non-decreasing and $\frac{1}{2}\E[X^2]$-Lipschitz on $\R_{\geq 0}$. Moreover, $\psi_{P_0}$ is strictly convex, if $P_0$ is not a Dirac measure.
%\end{proposition}
%\begin{proof}
%\end{proof}
%\\

Let us compute the mutual information and the MMSE for particular choices of prior distributions:
\begin{example}[Gaussian prior: $P_0 = \cN(0,1)$]\label{ex:gaussian_psi}
			In that case $\E[X|Y]$ is simply the orthogonal projection of $X$ on $Y$: 
			$$
			\E[X|Y] = \frac{\E[XY]}{\E[Y^2]} Y = \frac{\sqrt{\lambda}}{1+\lambda} Y.
			$$
			One deduces $\MMSE_{P_0}(\lambda) = \frac{1}{1+\lambda}$.
			Using~\eqref{eq:i_mmse}, we get $I(X;Y) = \frac{1}{2} \log(1+\lambda)$ and $\psi_{P_0}(\lambda) = \frac{1}{2}\big(\lambda - \log(1+\lambda)\big)$.
\end{example}

\begin{remark}[Worst-case prior]\label{rem:worst_case}
	Let $P_0$ be a probability distribution on $\R$ with unit second moment $\E_{P_0}[X^2] = 1$. By considering the estimator $\widehat{x} = \frac{\sqrt{\lambda}}{1+\lambda} Y$, one obtain $\MMSE_{P_0}(\lambda) \leq \frac{1}{1+ \lambda}$. We conclude:
	$$
	\sup_{P_0} \MMSE_{P_0}(\lambda) = \frac{1}{1+\lambda} \qquad \text{and} \qquad
	\inf_{P_0} \psi_{P_0}(\lambda) = \frac{1}{2}(\lambda - \log(1+\lambda)),
	$$
	where the supremum and infimum are both over the probability distributions that have unit second moment. The standard normal distribution $P_0 = \cN(0,1)$ achieves both extrema.
\end{remark}

\begin{example}[Rademacher prior: $P_0 = \frac{1}{2} \delta_{+1} + \frac{1}{2} \delta_{-1}$] We compute $\psi_{P_0}(\lambda) = \E \log \cosh(\sqrt{\lambda} Z + \lambda)- \frac{\lambda}{2}$ and $I(X;Y) = \lambda - \E \log \cosh(\sqrt{\lambda} Z + \lambda)$. The I-MMSE relation gives
			\begin{align*}
				\frac{1}{2} \MMSE(\lambda) 
				&= \frac{\partial}{\partial \lambda} I(X;Y)
				= 1 - \E \Big[\big(\frac{1}{2\sqrt{\lambda}}Z+1\big) \tanh\big(\sqrt{\lambda}Z + \lambda\big)\Big]
				\\
				&= 1 - \E \tanh(\sqrt{\lambda}Z + \lambda) - \frac{1}{2} \E \tanh'(\sqrt{\lambda}Z + \lambda)
				\\
				&= \frac{1}{2} - \E \tanh(\sqrt{\lambda}Z + \lambda) + \frac{1}{2} \E \tanh^2(\sqrt{\lambda}Z + \lambda)
			\end{align*}
			where we used Gaussian integration by parts. Since by the Nishimori property $\E \langle x X \rangle_{\lambda} = \E \langle x \rangle_{\lambda}^2$, one has $\E \tanh(\sqrt{\lambda}Z + \lambda) = \E \tanh^2(\sqrt{\lambda}Z + \lambda)$ and therefore $\MMSE(\lambda) = 1 - \E \tanh(\sqrt{\lambda}Z + \lambda)$.
\end{example}


\section{Bayes-optimal inference}
We introduce in this chapter some general properties of Bayes-optimal inference, that will be used repeatedly in the sequel.
Let us first define what we mean by \textit{Bayes-optimal inference}.

We consider a statistical problem where we would like to recover a signal vector $\bbf{X} \in \R^n$ from some observations $\bbf{Y} \in \R^m$.
We assume that $(\bbf{X},\bbf{Y})$ is drawn from some probability distribution $\mu$ over $\R^{n} \times \R^m$. Given a performance criterion, a Bayes-optimal estimator (or simply Bayes estimator) is an estimator of $\bbf{X}$ given $\bbf{Y}$ that achieves the best performance for this criterion. For instance if we measure the performance of an estimator $\what{\bbf{x}}$ by its mean square error $\MSE(\what{\bbf{x}}) = \E \| \bbf{X} - \what{\bbf{x}}(\bbf{Y}) \|^2$, then the Bayes-optimal estimator is simply the posterior mean $\what{\bbf{x}}^{\rm Bayes}(\bbf{Y}) = \E[\bbf{X}|\bbf{Y}]$.
\\

The goal of this chapter is to present some general properties of Bayes-optimal estimators. In Section~\ref{sec:nishimori} we introduce what we will call (according to the statistical physics terminology) the ``Nishimori identity'' which is nothing more than a rewriting of Bayes rule.
In Section~\ref{sec:performance} we will study the links between various natural performance metrics for estimators and show that they are in some sense equivalent.
In Sections~\ref{sec:i_mmse} we analyse the special case where $\bbf{Y} = \sqrt{\lambda}\bbf{X} + \bbf{Z}$, where $\lambda \geq 0$ and $\bbf{Z}$ is some Gaussian noise. This is the starting point of the study of the ``spiked'' matrix and tensor models. Finally we consider in Section~\ref{sec:rem} a simple example to illustrate the tools of this chapter.

\section{The Nishimori identity}\label{sec:nishimori}

In order to analyze Bayes-optimal estimators, we will need to examine the posterior distribution of $\bbf{X}$ given $\bbf{Y}$.
To do so we will often consider i.i.d.\ samples $\bbf{x}^{(1)}, \dots, \bbf{x}^{(k)}$ from the posterior distribution $P( \cdot \, | \, \bbf{Y} )$, independently of everything else. Such samples are called replicas. The (obvious) identity below (which is simply Bayes rule)
is named after the works of Nishimori \cite{nishimori1980exact,nishimori1981internal} on ``gauge-symmetric'' spin glasses.
It states that the planted solution $\bbf{X}$ behaves like a replica.

\begin{proposition}[Nishimori identity] \label{prop:nishimori}
	Let $(\bbf{X},\bbf{Y})$ be a couple of random variables on a polish space. Let $k \geq 1$ and let $\bbf{x}^{(1)}, \dots, \bbf{x}^{(k)}$ be $k$ i.i.d.\ samples (given $\bbf{Y}$) from the distribution $\P(\bbf{X}=\cdot \, | \, \bbf{Y})$, independently of every other random variables. Let us denote $\langle \cdot \rangle$ the expectation with respect to $\P(\bbf{X}=\cdot \, | \, \bbf{Y})$ and $\E$ the expectation with respect to $(\bbf{X},\bbf{Y})$. Then, for all continuous bounded function $f$
	$$
	\E \big\langle f(\bbf{Y},\bbf{x}^{(1)}, \dots, \bbf{x}^{(k)}) \big\rangle
	=
	\E \big\langle f(\bbf{Y},\bbf{x}^{(1)}, \dots, \bbf{x}^{(k-1)}, \bbf{X}) \big\rangle\,.
	$$
\end{proposition}
\begin{proof}
It is equivalent to sample the couple $(\bbf{X},\bbf{Y})$ according to its joint distribution or to sample first $\bbf{Y}$ according to its marginal distribution and then to sample $\bbf{X}$ conditionally to $\bbf{Y}$ from its conditional distribution $\P(\bbf{X}=\cdot \,|\,\bbf{Y})$. Thus the $(k+1)$-tuple $(\bbf{Y},\bbf{x}^{(1)}, \dots,\bbf{x}^{(k)})$ is equal in law to $(\bbf{Y},\bbf{x}^{(1)},\dots,\bbf{x}^{(k-1)},\bbf{X})$.
\end{proof}


\section{Performance measure and optimal estimators}\label{sec:performance}

We consider two random vectors $\bbf{X}$ and $\bbf{Y}$ that live respectively in $\R^n$ and $\R^m$. 
We assume (for simplicity) that $\|\bbf{X}\| = 1$ almost surely. 
As explained above, given the observations $\bbf{Y}$, our goal is to estimate $\bbf{X}$ with an estimator $\what{\bbf{x}}(\bbf{Y})$. 
In order to evaluate the performance of such an estimator, what criterion should we take?

The probably most natural way to characterize the performance of $\what{\bbf{x}}$ is by its mean-squared error:
$$
\MSE(\what{\bbf{x}}) = \E \| \bbf{X} - \what{\bbf{x}}(\bbf{Y}) \|^2.
$$
By Pythagorean theorem, we know that the optimal estimator which respect to this metric is the posterior mean $\what{\bbf{x}}(\bbf{Y}) = \E[\bbf{X}| \bbf{Y}]$ which achieves the minimal mean square error:
\begin{equation}\label{eq:def_mmse0}
	\MMSE(\bbf{X}|\bbf{Y}) \defeq \E \| \bbf{X} - \E[\bbf{X}|\bbf{Y}] \|^2.
\end{equation}

However, the MSE is not always an appropriate criterion. Indeed in many cases it is only possible to recover $\bbf{X}$ up to its sign: think for instance of the Spiked Wigner Model $\bbf{Y} = \bbf{X}\bbf{X}^{\sT} + {\rm Noise}$ with $\bbf{X} \sim \Unif(\mathbb{S}^{n-1})$. In such case, $\E[\bbf{X}|\bbf{Y}] = 0$: the best estimator in term of MSE does not even depend on the observations $\bbf{Y}$!
\\

For this kind of problems one should rather consider the correlation (also known as cosine similarity) \textit{in absolute value} between $\what{\bbf{x}}$ and the signal $\bbf{X}$:
\begin{equation}\label{eq:correlation_metric}
\sup_{\|\what{\bbf{x}}\| = 1} \E\big[|(\what{\bbf{x}}(\bbf{Y}); \bbf{X})|\big]
\qquad \text{or} \qquad
\sup_{\|\what{\bbf{x}}\| = 1} \E\big[(\what{\bbf{x}}(\bbf{Y}); \bbf{X})^2\big],
\end{equation}
where $(\cdot; \cdot)$ denotes the Euclidean inner product and where the suprema are taken over all estimators $\what{\bbf{x}}: \R^m \to \mathbb{S}^{n-1}$.
%The optimal estimator with respect to these metrics is however not obvious anymore!
\\

Let us introduce some notations. We will use the Gibbs Bracket $\langle \cdot \rangle$ to write expectations with respect to the posterior distribution of $\bbf{X}$ given $\bbf{Y}$:
$$
\langle f(\bbf{x}) \rangle = \E[f(\bbf{X})|\bbf{Y}],
$$
for all measurable function $f$ such that $f(\bbf{X})$ is integrable.
In particular, we will be interested by the $n \times n$ positive semi-definite (random) matrix:
\begin{equation}\label{eq:def_M}
\bbf{M} \defeq \langle \bbf{x} \bbf{x}^{\sT} \rangle = \E[\bbf{X}\bbf{X}^{\sT}|\bbf{Y}].
\end{equation}
$\bbf{M}$ is the Bayes-optimal estimator (in terms of mean square error) for estimating the matrix $\bbf{X} \bbf{X}^{\sT}$:
\begin{equation}\label{eq:def_matrix_mse}
\MMSE(\bbf{X}\bbf{X}^{\sT}|\bbf{Y}) = \E \| \bbf{X}\bbf{X}^{\sT} - \E[\bbf{X}\bbf{X}^{\sT}|\bbf{Y}] \|^2.
\end{equation}
An easy computation gives $\MMSE(\bbf{X}\bbf{X}^{\sT}|\bbf{Y}) = 1 - \E [\Tr(\bbf{M}^2)]$.
The matrix $\bbf{M}$ is also related to the second quantity of \eqref{eq:correlation_metric} through its largest eigenvalue $\lambda_{\rm max}(\bbf{M})$:
\begin{lemma}\label{lem:lambda_max}
	$$
	\sup_{\|\what{\bbf{x}}\| = 1} \E\big[(\what{\bbf{x}}(\bbf{Y}); \bbf{X})^2\big] = \E\big[\lambda_{\rm max}(\bbf{M})\big]
	$$
	and the optimal estimator for this metric is a unit eigenvector of $\bbf{M}$ associated to its largest eigenvalue $\lambda_{\rm max}(\bbf{M})$.
\end{lemma}
\begin{proof}
Let $\what{\bbf{x}}$ be an estimator of $\bbf{X}$. By the Nishimori identity (Proposition~\ref{prop:nishimori}), we have
	\begin{align*}
\E\big[(\what{\bbf{x}}(\bbf{Y}); \bbf{X})^2\big]
&=
\E\big[\what{\bbf{x}}(\bbf{Y})^{\sT} \bbf{X}\bbf{X}^{\sT} \what{\bbf{x}}(\bbf{Y})\big]
=
\E\big[\langle \what{\bbf{x}}(\bbf{Y})^{\sT} \bbf{x}\bbf{x}^{\sT} \what{\bbf{x}}(\bbf{Y}) \rangle\big]
=
\E\big[ \what{\bbf{x}}(\bbf{Y})^{\sT} \bbf{M} \what{\bbf{x}}(\bbf{Y}) \big],
	\end{align*}
	the lemma follows.
\end{proof}
\\

%We know now what are the optimal estimators corresponding to
%$\MMSE(\bbf{X}\bbf{X}^{\sT}|\bbf{Y})$ and $\sup_{\|\what{\bbf{x}}\| = 1} \E\big[(\what{\bbf{x}}(\bbf{Y}); \bbf{X})^2\big]$, and thanks to the inequality $\lambda_{\rm max}(\bbf{M})^2 \leq \Tr(\bbf{M}^2)$ we are able to compare the two performance measures.

%However, this is not the case for the criteria $\E\left[\left|(\what{\bbf{x}}(\bbf{Y}); \bbf{X})\right|\right]$, for which we do not know the optimal estimator. 
%Further, assume that we aim at estimating the rank-one matrix $\bbf{X}\bbf{X}^{\sT}$, should we use $\bbf{M}$ (which will typically not be rank-one) or $\what{\bbf{v}} \what{\bbf{v}}^{\sT}$ where $\what{\bbf{v}}$ is the leading unit eigenvector of $\bbf{M}$?

Lemma~\ref{lem:lambda_max} tells us that the top unit eigenvector $\what{\bbf{v}}$ of $\bbf{M}$ maximizes $\E\left[(\what{\bbf{x}}(\bbf{Y}); \bbf{X})^2\right]$.
In the following, we will show that under a simple condition (that will hold for the models we consider in this manuscript), the estimator $\what{\bbf{v}}$ is ``asymptotically optimal'' (in the limit of large dimension) for the two metrics \eqref{eq:correlation_metric} and $\lambda_{\rm max} \what{\bbf{v}} \what{\bbf{v}}^{\sT}$ is optimal for the estimation of $\bbf{X}\bbf{X}^{\sT}$ in terms of mean square error.
\\

To introduce this condition and the asymptotic limit, we need to consider to a sequence of inference problems. We assume that for all $n \geq 1$ we have two random vectors $\bbf{X}_{[n]}$ and $\bbf{Y}_{[n]}$ respectively in $\mathbb{S}^{n-1}$ and $\R^{m_n}$, for some sequence $(m_n)_{n \geq 1}$. Our goal is again to estimate $\bbf{X}_{[n]}$ from the observation of $\bbf{Y}_{[n]}$ when $n$ is very large: we would like for instance to compute the limits of \eqref{eq:correlation_metric} and \eqref{eq:def_matrix_mse} as $n \to \infty$.
Moreover, we would like to know which estimators are ``asymptotically optimal'', i.e.\ whose performance reach in the $n \to \infty$ limit the optimal one.
In the following, in order to simplify the notations, we will write $\bbf{X}$ and $\bbf{Y}$ instead of $\bbf{X}_{[n]}$ and $\bbf{Y}_{[n]}$.


\begin{proposition}\label{prop:lambda_max}
	Let us denote by $G_n$ the posterior distribution of $\bbf{X}$ given $\bbf{Y}$. Notice that $G_n$ is a random probability distributions on $\mathbb{S}^{n-1}$.
	Assume that there exists $q \in [0,1]$ such that
	for $\bbf{x}^{(1)},\bbf{x}^{(2)} \iid G_n$ we have
	\begin{equation}\label{eq:rs_condition}
	\big|\big(\bbf{x}^{(1)}; \bbf{x}^{(2)}\big)\big| \xrightarrow[n \to \infty]{\rm (d)} q.
	\end{equation}
	Then $\Tr(\bbf{M}^2) \xrightarrow[n \to \infty]{\rm (d)} q^2$ and
	$$
	\lambda_{\rm max}(\bbf{M}) \xrightarrow[n \to \infty]{\rm (d)} q.
	$$
\end{proposition}
\begin{proof}
	%Since $\|\bbf{X}\| = 1$ almost surely we have $\Tr(\bbf{M}) = \langle \|\bbf{x}\|^2 \rangle = 1$. 
	Let us compute $\Tr(\bbf{M}^2) = \Tr \big(\langle \bbf{x} \bbf{x}^{\sT} \rangle \langle \bbf{x} \bbf{x}^{\sT} \rangle \big) = \langle (\bbf{x}^{(1)}; \bbf{x}^{(2)})^2 \rangle \xrightarrow[n \to \infty]{} q^2$,
	by assumption. 
	If $q =0$, then the result is obvious since $\lambda_{\rm max}(\bbf{M})^2 \leq \Tr(\bbf{M}^2)$.

	Notice that $\Tr(\bbf{M}^3) \leq \lambda_{\rm max}(\bbf{M}) \Tr(\bbf{M}^2)$, so it suffices to show that
$$
{\rm Tr}(\bbf{M}^3) = \big\langle (\bbf{x}^{(1)};\bbf{x}^{(2)}) (\bbf{x}^{(2)};\bbf{x}^{(3)}) (\bbf{x}^{(3)};\bbf{x}^{(1)}) \big\rangle
\xrightarrow[n \to \infty]{} q^3,
$$
for $\bbf{x}^{(1)},\bbf{x}^{(2)},\bbf{x}^{(3)} \iid G_n$. This follows from Lemma~\ref{lem:positivity} below.
\end{proof}
\\

\begin{lemma}\label{lem:positivity}
	Under the assumptions of Proposition~\ref{prop:lambda_max}, we have for $\bbf{x}^{(1)},\bbf{x}^{(2)},\bbf{x}^{(3)} \iid G_n$,
$$
(\bbf{x}^{(1)};\bbf{x}^{(2)}) (\bbf{x}^{(2)};\bbf{x}^{(3)}) (\bbf{x}^{(3)};\bbf{x}^{(1)}) \xrightarrow[n \to \infty]{\rm (d)} q^3.
$$
\end{lemma}
Lemma~\ref{lem:positivity} will be proved in Appendix~\ref{sec:proof_positivity}.
From Proposition~\ref{prop:lambda_max} we deduce the main result of this section:

\begin{proposition}\label{prop:lim_error_metrics}
	Let $\what{\bbf{v}}$ be a leading unit eigenvector of $\bbf{M}$ (which is defined by \eqref{eq:def_M}).
	Under the assumptions of Proposition~\ref{prop:lambda_max}, we have
	\begin{equation}\label{eq:lim_leading}
	| (\what{\bbf{v}}; \bbf{X}) | \xrightarrow[n \to \infty]{\rm (d)} \sqrt{q}.
	\end{equation}
Further $\lim\limits_{n \to \infty} \MMSE(\bbf{X}\bbf{X}^{\sT}|\bbf{Y}) = 1 - q^2$, 
\begin{equation}\label{eq:lim_correlation_metric}
	\lim_{n \to \infty} \sup_{\|\what{\bbf{x}}\| = 1} \E\big[|(\what{\bbf{x}}(\bbf{Y}); \bbf{X})|\big] = \sqrt{q},
	\qquad \text{and} \qquad
\lim_{n \to \infty} \sup_{\|\what{\bbf{x}}\| = 1} \E\big[(\what{\bbf{x}}(\bbf{Y}); \bbf{X})^2\big] = q.
\end{equation}
\end{proposition}
\begin{proof}
	Let us abbreviate $\lambda_{\rm max} \defeq \lambda_{\rm max}(\bbf{M})$.
	By Lemma~\ref{lem:lambda_max}  and Proposition~\ref{prop:lambda_max} we have
	\begin{equation}\label{eq:lim2v}
	\E\big[(\what{\bbf{v}}; \bbf{X})^2\big] = \E[\lambda_{\rm max}] \xrightarrow[n \to \infty]{} q.
	\end{equation}
	Hence if $q=0$, the Proposition follows easily. Assume now that $q > 0$.
	Using Pythagorean Theorem and the Nishimori identity (Proposition~\ref{prop:nishimori}) we get
	\begin{align*}
		\E \| \bbf{X}^{\otimes 4} - \lambda_{\rm max}^2 \what{\bbf{v}}^{\otimes 4} \|^2
		&\geq
		\E \| \bbf{X}^{\otimes 4} - \langle \bbf{x}^{\otimes 4}\rangle \|^2
		\\
		&=
		1 + \E \big[ \big( \langle \bbf{x}^{\otimes 4}\rangle ; \langle \bbf{x}^{\otimes 4}\rangle \big) \big]
		- 2 \E \big[\big( \bbf{X}^{\otimes 4} ; \langle \bbf{x}^{\otimes 4}\rangle\big) \big]
		\\
		&= 1 - \E \left[ \left\langle (\bbf{x}^{(1)}; \bbf{x}^{(2)})^4 \right\rangle \right]
		\xrightarrow[n \to \infty]{} 1 - q^4,
	\end{align*}
	where the last limit follows from the assumption \eqref{eq:rs_condition}. Since
	$$
		\E \| \bbf{X}^{\otimes 4} - \lambda_{\rm max}^2 \what{\bbf{v}}^{\otimes 4} \|^2
		= 1 + \E\big[\lambda_{\rm max}^4\big] - 2 \E\big[\lambda_{\rm max}^2 (\bbf{X};\what{\bbf{v}})^4\big],
	$$
	using Proposition~\ref{prop:lambda_max}, we deduce (recall that we assumed $q >0$) that $\limsup_{n \to \infty} \E\big[(\what{\bbf{v}}; \bbf{X})^4\big] \leq q^2$. Together with \eqref{eq:lim2v} this gives that $|(\what{\bbf{v}}; \bbf{X})| \xrightarrow[n \to \infty]{} \sqrt{q}$.

	The next point is a consequence of Proposition~\ref{prop:lambda_max} because
	$\MMSE(\bbf{X}\bbf{X}^{\sT}|\bbf{Y}) = 1 - \E [\Tr(\bbf{M}^2)]$. To prove \eqref{eq:lim_correlation_metric} simply notice that
	$$
	\E\big[|(\what{\bbf{v}};\bbf{X})|\big]^2
	\leq \sup_{\|\what{\bbf{x}}\| = 1} \E\big[|(\what{\bbf{x}}(\bbf{Y}); \bbf{X})|\big]^2
	\leq
	\sup_{\|\what{\bbf{x}}\| = 1} \E\big[(\what{\bbf{x}}(\bbf{Y}); \bbf{X})^2\big] = 
	\E[\lambda_{\rm max}],
	$$
	which proves \eqref{eq:lim_correlation_metric} using \eqref{eq:lim_leading} and Proposition~\ref{prop:lambda_max}. 
\end{proof}
\\

From Proposition~\ref{prop:lim_error_metrics}, we deduce that the estimator $\what{\bbf{A}} \defeq \lambda_{\rm max}(\bbf{M}) \, \what{\bbf{v}}\what{\bbf{v}}^{\sT}$ achieves asymptotically the minimal mean square error for the estimation of $\bbf{X}\bbf{X}^{\sT}$:
$$
\lim_{n \to \infty} \E \| \bbf{X} \bbf{X}^{\sT} - \what{\bbf{A}} \|^2 = 1-q^2.
$$

\begin{remark}\label{rem:cosine}
For simplicity we assumed in this section that $\|\bbf{X}\|^2 = 1$ almost surely. However we will need to work in the next chapters under a slightly weaker condition, namely $\|\bbf{X}\|^2 \xrightarrow[n \to \infty]{} 1$. It is not difficult to modify the proofs of this section to see that Lemma~\ref{lem:lambda_max}, Proposition~\ref{prop:lambda_max}, Lemma~\ref{lem:positivity} and Proposition~\ref{prop:lim_error_metrics} still hold, provided that $\|\bbf{X}\|^2 \xrightarrow[n \to \infty]{} 1$ for the Wasserstein distance of order $4$ (i.e.\ $\|\bbf{X}\|^2 \xrightarrow[n \to \infty]{} 1$ in distribution and $\E \|\bbf{X}\|^8 \xrightarrow[n \to \infty]{} 1$).
\end{remark}





\section{A warm-up: the ``needle in a haystack'' problem}\label{sec:rem}

In order to illustrate the results seen in the previous sections, we study now a very simple inference model.
Let $(e_1,\dots,e_{2^n})$ be the canonical basis of $\R^{2^n}$. Let $\sigma_0 \sim \Unif(\{1,\dots,2^n\})$ and define $\bbf{X} = e_{\sigma_0}$ (i.e.\ $\bbf{X}$ is chosen uniformly over the canonical basis of $\R^{2^n}$).
Suppose here that we observe:
$$
\bbf{Y} = \sqrt{\lambda n} \bbf{X} + \bbf{Z} \,,
$$
where $\bbf{Z} = (Z_1, \dots, Z_{2^n}) \iid \cN(0,1)$, independently from $\sigma_0$. The goal here is to estimate $\bbf{X}$ or equivalently to find $\sigma_0$.
The posterior distribution reads:
\begin{align*}
\P(\sigma_0 = \sigma | \bbf{Y}) = 
\P(\bbf{X} = e_{\sigma} | \bbf{Y}) &= 
\frac{1}{\cZ_{n}(\lambda)} 2^{-n} \exp \Big( \sqrt{\lambda n} e_{\sigma}^{\sT} \bbf{Y} - \frac{\lambda n}{2} \|e_{\sigma}\|^2 \Big)
\\
&=
\frac{1}{\cZ_{n}(\lambda)} 2^{-n} \exp \Big( \sqrt{\lambda n}Z_{\sigma} + \lambda n \1(\sigma = \sigma_0) - \frac{\lambda n}{2} \Big) ,
\end{align*}
where $\cZ_n(\lambda)$ is the partition function
$$
\cZ_{n}(\lambda) = \frac{1}{2^n} \sum_{\sigma=1}^{2^n} \exp \Big( \sqrt{\lambda n}Z_{\sigma} + \lambda n \1(\sigma = \sigma_0) - \frac{\lambda n}{2} \Big) \,.
$$
We will be interested in computing the free energy $F_n(\lambda) = \frac{1}{n} \E \log \cZ_n(\lambda)$ in order to deduce then the minimal mean squared error using the I-MMSE relation~\eqref{eq:i_mmse} presented in the previous section. 

Although its simplicity, this model is interesting for many reasons. 
First, it is one of the simplest statistical model for which one observes a phase transition.
Second it is the ``planted'' analog of the random energy model (REM) introduced in statistical physics by Derrida~\cite{derrida1980random,derrida1981random}, for which the free energy reads $\frac{1}{n} \E \log \sum_{\sigma}\frac{1}{2^n} \exp\big(\sqrt{\lambda n} Z_{\sigma}\big)$. Third, as we will see in Section~\ref{sec:tensor_large_order}, this model correspond to the ``large order limit'' of a rank-one tensor estimation model. 

We start by computing the limiting free energy:

\begin{theorem}
$$
\lim_{n\to \infty} F_n(\lambda) = 
\begin{cases}
	0 & \text{if} \quad \lambda \leq 2 \log 2 \,, \\
	\frac{\lambda}{2} - \log(2)& \text{if} \quad \lambda \geq 2 \log 2 \,.
\end{cases}
$$
\end{theorem}
\begin{proof}
	Using Jensen's inequality
	\begin{align*}
	F_n(\lambda) 
	&\leq \frac{1}{n} \E \log \E \left[ \cZ_n(\lambda) \middle| \sigma_0, Z_{\sigma_0} \right]
	= \frac{1}{n} \E \log \left(1 - \frac{1}{2^n} + e^{\sqrt{\lambda n} Z_{\sigma_0} + \frac{\lambda n}{2} - \log(2)n}\right)
	\\
	&\leq \frac{1}{n} \E \log \left(1 + e^{\frac{\lambda n}{2} - \log(2)n}\right) + \sqrt{\frac{\lambda}{n}}
	\ \xrightarrow[n \to \infty]{}  \
	\begin{cases}
		0 & \text{if} \quad \lambda \leq 2 \log(2)\,, \\
		\frac{\lambda}{2} - \log(2) & \text{if} \quad \lambda \geq 2 \log(2) \,.
	\end{cases}
	\end{align*}
	$F_n$ is non-negative since $F_n(0) = 0$ and $F_n$ is non-decreasing. We have therefore $F_n(\lambda) \xrightarrow[n \to \infty]{} 0$ for all $\lambda \in [0,2\log(2)]$.
	We have also, by only considering the term $\sigma = \sigma_0$:
	$$
	F_n(\lambda) \geq \frac{1}{n} \E \log \left(\frac{e^{\sqrt{\lambda n}Z_{\sigma_0} + \frac{\lambda n}{2}}}{2^n}\right) = \frac{\lambda}{2} - \log(2) \,.
	$$
	We obtain therefore that $F_n(\lambda ) \xrightarrow[n \to \infty]{} \frac{\lambda}{2} - \log(2)$ for $\lambda \geq 2 \log(2)$.
\end{proof}
\\

Using the I-MMSE relation~\eqref{eq:i_mmse}, we deduce the limit of the minimum mean squared error $\MMSE_n(\lambda) =  \min_{\widehat{\theta}} \E \| \bbf{X} - \widehat{\theta}(\bbf{Y}) \|^2$:
$$
\MMSE_n(\lambda) = \E \|\bbf{X}\|^2 - 2 F_n'(\lambda) = 1 - 2 F_n'(\lambda) \,.
$$
$F_n$ is a convex function of $\lambda$, thus (see Proposition~\ref{prop:deriv_convex}) its derivative converges to the derivative of its limit at each $\lambda$ at which the limit is differentiable, i.e.\ for all $\lambda \in (0,+\infty) \setminus \{2 \log(2) \}$. We obtain therefore that for all $\lambda >0$,
\begin{itemize}
	\item if $\lambda < 2 \log(2)$, then $\MMSE_n(\lambda) \xrightarrow[n \to \infty]{} 1$: one can not recover $\bbf{X}$ better than a random guess.
	\item if $\lambda > 2 \log(2)$, then $\MMSE_n(\lambda) \xrightarrow[n \to \infty]{} 0$: one can recover $\bbf{X}$ perfectly.
\end{itemize}
Of course, the result we obtain here is (almost) trivial since the maximum likelihood estimator 
$$
\widehat{\sigma}(\bbf{Y}) = \argmax_{1 \leq \sigma \leq 2^n} \, Y_{\sigma}
$$
of $\sigma_0$ is easy to analyze. Indeed, $\max_{\sigma} Z_{\sigma} \simeq \sqrt{2 \log(2)n}$ with high probability so that the maximum likelihood estimator recovers perfectly the signal for $\lambda > 2 \log(2)$ with high probability.


\section{Appendix}
\subsection{Proof of Proposition~\ref{prop:mmse_cont}} \label{sec:proof_mmse_cont}
	We start by proving that $\MMSE$ is continuous at $\lambda = 0$.
	Let $\lambda\geq0$ and consider $\bbf{Y},\bbf{X},\bbf{Z}$ as given by~\eqref{eq:inference}.
	%We have obviously $\MMSE(\lambda) \leq \E \big\| \bbf{X} - \E[\bbf{X}] \big\|^2$.
	%Let $c >0$ such that $\P(\|\bbf{X}\| \leq c)>0$ and define the random variable $B_c = \1(\|\bbf{X}\| \leq c)$.
	%\begin{equation}\label{eq:lower_bound_mmse}
		%\MMSE(\lambda) 
		%%&= \E \big\| \bbf{X} - \E [\bbf{X} | \bbf{Y}] \big\|^2
		%\geq \E \big\| \bbf{X} - \E [\bbf{X} | \bbf{Y},B_c] \big\|^2
		%\geq \E \left[  B_c \big\| \bbf{X} - \E [\bbf{X} | \bbf{Y},B_c] \big\|^2\right]\,.
		%%\\
		%%&= \E \left[ \big\| \bbf{X} B_c - \E [\bbf{X} B_c | \bbf{Y},B_c] \big\|^2\right]
	%\end{equation}
	By dominated convergence one has almost surely that 
	$$
	\E[\bbf{X} | \bbf{Y}]
	= 
	\frac{\int dP_X(\bbf{x}) \bbf{x} e^{-\frac{1}{2}\| \sqrt{\lambda} \bbf{x} - \bbf{Y}\|^2}}{\int dP_X(\bbf{x}) e^{-\frac{1}{2}\| \sqrt{\lambda} \bbf{x} - \bbf{Y}\|^2}}
	\xrightarrow[\lambda \to 0]{} \E[\bbf{X}] \,.
	$$
	Then by Fatou's Lemma we get
	$$
	\liminf_{\lambda \to 0} \MMSE(\lambda) \geq \E \Big[\liminf_{\lambda \to 0}
		\big\| \bbf{X} - \E[\bbf{X}| \bbf{Y}] \big\|^2
	\Big] = \E \big\| \bbf{X} - \E[\bbf{X}] \big\|^2 \,.
	$$
	%$$
	%B_c \E [\bbf{X} | \bbf{Y},B_c]
	%=
	%B_c
	%\frac{\int_{\|\bbf{x}\| \leq c} dP_X(\bbf{x}) \bbf{x} e^{H_{\lambda,\bbf{Y}}(\bbf{x})}}{\int_{\|\bbf{x}\| \leq c} dP_X(\bbf{x}) e^{H_{\lambda,\bbf{Y}}(\bbf{x})}}
	%\xrightarrow[\lambda \to 0]{} B_c \P(\|\bbf{X}\| \leq c)^{-1}\E [\1(\|\bbf{X}\| \leq c)\bbf{X} ] \,,
	%$$
	%and therefore another application of the dominated convergence theorem gives:
	%$$
	%\E \left[  B_c \big\| \bbf{X} - \E [\bbf{X} | \bbf{Y},B_c] \big\|^2\right]
	%\xrightarrow[\lambda \to 0]{}
	%\E \left[ \1(\|\bbf{X}\|\leq c) \Big\| \bbf{X} - \P(\|\bbf{X}\| \leq c)^{-1} \E [\1(\|\bbf{X}\|\leq c)\bbf{X}] \Big\|^2\right] \,.
	%$$
	%Again, by the dominated convergence theorem, the right-hand side converges to $\E \| \bbf{X} -\E[\bbf{X}]\|^2$ as $c \to +\infty$. Going back to~\eqref{eq:lower_bound_mmse}, we obtain that
	%$$
	%\liminf_{\lambda \to 0} \MMSE(\lambda) \geq \E \| \bbf{X} -\E[\bbf{X}]\|^2\,,
	%$$
	Combining this with the bound $\MMSE(\lambda) \leq \E \| \bbf{X} - \E[\bbf{X}]\|^2$ gives $\MMSE(\lambda) \xrightarrow[\lambda \to 0]{} \E \| \bbf{X} -\E[\bbf{X}]\|^2$. This proves that the $\MMSE$ is continuous at $\lambda=0$.
	\\

	Let us now prove that the $\MMSE$ is continuous on $\R_{\geq 0}^*$. We need here a technical lemma:
	\begin{lemma}\label{lem:ui1}
		For all $\lambda > 0$, $p \geq 1$
		$$
		\E \| \bbf{X} - \langle \bbf{x} \rangle_{\lambda} \|^{2p} \leq \frac{2^p (2p !)}{\lambda ^{p} p!} n^{p+1} \,.
		$$
	\end{lemma}
	\begin{proof}
		We reproduce here the proof from~\cite{guo2011estimation}, Proposition~5. We start with the equality
		$$
		\sqrt{\lambda} \left(\bbf{X} - \langle \bbf{x} \rangle_{\lambda}\right)
		=
		\sqrt{\lambda} \bbf{X} - \E [ \sqrt{\lambda} \bbf{X} | \bbf{Y} ]
		=
		\bbf{Y} - \bbf{Z} - \E [ \bbf{Y} - \bbf{Z} | \bbf{Y} ]
		=
		\E [ \bbf{Z} | \bbf{Y} ] -\bbf{Z} \,.
		$$
		We have therefore
		\begin{align*}
			\E \| \bbf{X} - \langle \bbf{x} \rangle_{\lambda} \|^{2p} 
			&= \frac{1}{\lambda ^{p}} 
			\E \big\| \E [ \bbf{Z} | \bbf{Y} ] -\bbf{Z}\big\|^{2p}
			\leq 
			\frac{2^{2p-1}}{\lambda ^{p}} 
			\E \big[  \|\E [ \bbf{Z} | \bbf{Y} ]\|^{2p} +\|\bbf{Z}\|^{2p} \big]
			\leq
			\frac{2^{2p}}{\lambda ^{p}} 
			\E \|\bbf{Z}\|^{2p} \,.
		\end{align*}
		It remains to bound
		$$
		\E \|\bbf{Z}\|^{2p}
		\leq n^p \E \left[\sum_{i=1}^n Z_i^{2p} \right]
		= n^{p+1} \frac{(2p)!}{2^p p!} \,.
		$$
	\end{proof}
	\\

	Let $\lambda_0 > 0$. The family of random variables $\big(\| \bbf{X} - \langle \bbf{x}\rangle_{\lambda}\|^2\big)_{\lambda \geq \lambda_0}$ is bounded in $L^2$ by Lemma~\ref{lem:ui1} and is therefore uniformly integrable. The function $\lambda \mapsto \| \bbf{X} - \langle \bbf{x}\rangle_{\lambda}\|^2$ is continuous on $[\lambda_0,+\infty)$, the uniform integrability ensures then that $\MMSE: \lambda \mapsto  \E\| \bbf{X} - \langle \bbf{x}\rangle_{\lambda}\|^2$ is continuous over $[\lambda_0,+\infty)$. This is valid for all $\lambda_0 >0$: we conclude that $\MMSE$ is continuous over $(0,+\infty)$.

\end{document}
